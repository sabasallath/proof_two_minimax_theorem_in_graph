\documentclass{article}
\usepackage{packages}
\usepackage{commands}
\graphicspath{{img/}}

%*************************************************************
\title{Explanation of "On Two Minimax Theorems in Graph"}
\author{Daniel Blévin\\George ??????}
\date\today

%*************************************************************
\begin{document}
\maketitle
\newpage
\tableofcontents
\newpage
%************************************************************* THEOREM
\newtheorem{edmonds_theorem}{THEOREM}

%************************************************************* 1
\section{Definitions}

\subsection{Edge-disjoint paths}
Two paths are edge-disjoint if they do not have any internal edge in common.

\subsection{Branching :}

% todo : merge in one definition (later)

\paragraph{Branching One}
In a directed graph $D = (V, A)$ a branching is a set of arcs not containing 
circuits s.t. each node of $D$ is entered at most by one arc in $A'$. 
(So the arc in $A'$ make up a forest). % A or A' ???

\paragraph{Branching Two} {
A branching $B$ is an arc set in a digraph $D$ that is a forest such that every node of $D$ 
is the head of at most one arc of $B$. 
A branching that is a tree is called an arborescence.
A branching that is a spanning tree of $D$ is called a spanning arborescence of $D$. 
Clearly, in a spanning arborescence B of $D$ every node of $D$ is the head of one arc of $B$ 
except for one node. 
This node is called the root of $B$. 
If $r$ is the root of arborescence $B$ we also say that $B$ is rooted at $r$ or $r-$rooted.
}

\paragraph{R-branching}
For a subset $R \subset V$, an $r-$branching in $G$ is a spanning forest $B \subset G$ in 
which all vertices of $V - R$ have outdegree precisely $1$.

\paragraph{r-branching}
When $R$ just consists of a single vertex $r$, we refer to $B$ as an $r-$branching.

\subsection{k-connected}
A graph $G$ (digraph $D$) is called $k-$connected ($k-$diconnected) if every pair
s, t of nodes is connected by at least $k$ $[s, t]-$paths ($(s, t)-$dipaths) whose sets of
internal nodes are mutually disjoint.

\subsection{a-cut}
A $a-$cut of $G$ determined by a set $S \subset V(G)$ is the set of
edges going from $S$ to $V(G) - S$. 
It will be denoted by $\Delta_G(S)$.
We also set that $\delta_G(S) = |\Delta_G(S)|$.

\subsection{r-Cuts and r-Arborescences}

Consider a connected digraph $(N, A)$ with $r \in N$ and nonnegative
integer arc lengths $l_a$ for $a \in A$. An $r-$arborescence is a minimal arc
set that contains an $rv-$dipath for every $v \in N$. 
It follows that an $r-$arborescence contains $|N| - 1$ arcs forming a spanning tree and each
node of $N - {r}$ is entered by exactly one arc.
The minimal transversals of the clutter of r-arborescences are called $r-$cuts.

\subsection{A note on orientation}
A rooted tree is a tree in which one vertex has been designated the root.
The edges of a rooted tree can be assigned a natural orientation, either away from 
or towards the root, in which case the structure becomes a directed rooted tree.
When a directed rooted tree has an orientation away from the root,
 it is called an arborescence, branching, or out-tree;
 when it has an orientation towards the root, 
 it is called an anti-arborescence or in-tree.

%************************************************************* 2
\section{Definitions examples}

\subsection{Branchings examples}

Taking the next figure as exemple :
\insertTikzFigure{1}{branchingExemple_base}{A graph}{tikz:branchingExemple_base}


\subsubsection{Simple branching}

Not yet.


\subsubsection{Edge disjoint branching}

Decomposition of the exemple figure in 3 branching routed respectively at $a, c, d, e$.
\insertTikzFigure{1}{branchingExemple_disjoint}{A graph}{tikz:branchingExemple_disjoint}

\subsubsection{Edge disjoint $a-$routed branching}

\insertTikzFigure{1}{branchingExemple_arouted}{A graph}{tikz:branchingExemple_arouted}


\subsection{Cuts examples}

\subsubsection{General cut}

\subsubsection{$a-$cut}


\section{Theorems}

\begin{edmonds_theorem}[Edmonds]
The maximum number of edge-disjoint branchings (rooted at $a$) 
equals the minimum number of edges in $a-$cuts.
\end{edmonds_theorem}

\end{document}

