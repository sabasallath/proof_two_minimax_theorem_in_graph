\documentclass{article}
\usepackage{packages}
\usepackage{commands}
\graphicspath{{img/}}

%*************************************************************
\title{Explanation of "On Two Minimax Theorems in Graph"}
\author{Daniel Blévin\\George Apaaboah}
\date\today

%*************************************************************
\begin{document}
\maketitle
%\newpage
\tableofcontents
%\newpage
%************************************************************* THEOREM
\newtheorem{edmonds_theorem}{THEOREM}

%************************************************************* 1
\section{Theorems}

\begin{edmonds_theorem}[Edmonds]
The maximum number of edge-disjoint branchings (rooted at $a$) 
equals the minimum number of edges in $a-$cuts.
\end{edmonds_theorem}

%************************************************************* 2
\section{Definitions}

\subsection*{Branching and a-cut (from the paper)}

\paragraph{branching}
In a directed graph $D = (V, A)$ a branching is a set of arcs not containing 
circuits s.t. each node of $D$ is entered at most by one arc in $A'$. 
(So the arc in $A'$ make up a forest). % A or A' ???

\paragraph{a-cut}
A $a-$cut of $G$ determined by a set $S \subset V(G)$ is the set of
edges going from $S$ to $V(G) - S$. 
It will be denoted by $\Delta_G(S)$.
We also set that $\delta_G(S) = |\Delta_G(S)|$.


\subsection*{Branching and a-cut (from books)}
\paragraph{branching}
A branching $B$ is an arc set in a digraph $D$ that is a forest such that every node of $D$ 
is the head of at most one arc of $B$. 
A branching that is a tree is called an arborescence.
A branching that is a spanning tree of $D$ is called a spanning arborescence of $D$. 
Clearly, in a spanning arborescence B of $D$ every node of $D$ is the head of one arc of $B$ 
except for one node. 
This node is called the root of $B$. 
If $r$ is the root of arborescence $B$ we also say that $B$ is rooted at $r$ or $r-$rooted.

\paragraph{r-arborescence}
An arborescence in a digraph $D = (V, A)$ is a set $A'$ of arcs making up a spanning tree 
such that each node of $D$ is entered by at most one arc in $A'$. 
It follows that there is exactly one node r that is not
entered by any arc of $A'$. This node is called the root of $A'$, and $A'$ is called
rooted in $r$, or an $r-$arborescence.

\paragraph{r-cut}
An r-cut in a digraph is an edge set $\delta^+(X)$ for some $X \subset V(G)$ with $r \in X$.

\paragraph{edge-disjoint branchings}
Two branching are edge-disjoint if they do not have any internal edge in common.

\paragraph{A note on orientation}
A rooted tree is a tree in which one vertex has been designated the root.
The edges of a rooted tree can be assigned a natural orientation, either away from 
or towards the root, in which case the structure becomes a directed rooted tree.
When a directed rooted tree has an orientation away from the root,
it is called an arborescence, branching, or out-tree;
when it has an orientation towards the root, 
it is called an anti-arborescence or in-tree.


% \subsection*{Alternative definition (from book 2)}

% Consider a connected digraph $(N, A)$ with $r \in N$ and non-negative
% integer arc lengths $l_a$ for $a \in A$. 
% An $r-$arborescence is a minimal arc set that contains an $rv-$dipath for every $v \in N$. 
% It follows that an $r-$arborescence contains $|N| - 1$ arcs forming a spanning tree and each
% node of $N - {r}$ is entered by exactly one arc.
% The minimal transversals of the clutter of r-arborescences are called $r-$cuts.
% \\\\
% We define a clutter C to be a family $E(C)$ of subsets of a finite ground
% set $V(C)$ with the property that $S_1 \nsubseteq S_2$ for all distinct $S_1, S_2 \in E(C)$.
% $V(C)$ is called the set of vertices and $E(C)$ the set of edges of $C$.

%************************************************************* 2
\section{Examples}

\subsection{Branching}

\subsubsection{Simple branching}
\begin{minipage}{.5\textwidth}
\centering
\insertTikzFigure{1}{branchingExemple_base}
{A digraph $D=(A, V)$}
{tikz:branchingExemple_base}
\end{minipage}%
\begin{minipage}{.5\textwidth}
\centering
\insertTikzFigure{1}{branchingExemple_simple}
{A branching of $D$ routed at $a$.}
{tikz:branchingExemple_simple}
\end{minipage}


\subsubsection{Edge disjoint branching}

\insertTikzFigure{1}{branchingExemple_disjoint}
{Two edge-disjoint branching routed respectively at $a$ and $d$.}
{tikz:branchingExemple_disjoint}

\subsubsection{Edge disjoint a-routed branching}

\insertTikzFigure{1}{branchingExemple_arouted}
{Two edge-disjoint $a$-routed branching.}
{tikz:branchingExemple_arouted}

\subsection{a-cut exemples}

\insertTikzFigure{1}{cutExemple_base}
{A digraph $D=(A, V)$.}
{tikz:cutExemple_base}

\insertTikzFigure{1}{cutExemple_cut1}
{An a-cut of $D$,\\ $S = \{a\}$,\\ $\Delta_D(S)=\{ab, ad\}$,\\ $\delta_D(S) = 2$.}
{tikz:cutExemple_cut1}

\insertTikzFigure{1}{cutExemple_cut2}
{An minimal a-cut of $D$,\\ $S = \{a, b, c, d, f\}$,\\ $\Delta_D(S)=\{be\}$,\\ $\delta_D(S) = 1$.}
{tikz:cutExemple_cut1}

\insertTikzFigure{1}{cutExemple_cut3}
{An minimal a-cut of $D$,\\ $S = \{a, b, d, e, f\}$,\\ $\Delta_D(S)=\{bc\}$,\\ $\delta_D(S) = 1$.}
{tikz:cutExemple_cut1}

\end{document}
    
    