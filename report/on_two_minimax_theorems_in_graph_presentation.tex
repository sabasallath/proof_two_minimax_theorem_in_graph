\documentclass[8pt]{beamer}
\usepackage{packages}
\usepackage{commands}
\usepackage{beamer_options}
\graphicspath{{img/}}
% Magic comment for latex studio
% !TeX root = ./on_two_minimax_theorems_in_graph_presentation.tex
% !TEX program = xelatex

%*************************************************************
\title{On Two Minimax Theorems in Graph}
\author{Daniel Blévin - George Apaaboah}
\date\today

%*************************************************************
\begin{document}
\maketitle
\begin{frame}
\tableofcontents
\end{frame}

%************************************************************* THEOREM
\newtheorem{edmonds_theorem}{THEOREM}
\newtheorem{Lucchesi_and_Younger}{THEOREM}
\newtheorem{ly_lemma}{LEMMA}
\newtheorem{hypergraph}{Theorem}

%=============================================================
\section{First theorem}
%-------------------------------------------------------------
\subsection{Definitions}
%*************************************************************
\begin{frame}
\frametitle{Branching}

\textbf{branching (r-arborescence) :}
In a directed graph $D = (V, A)$ a branching is a set of arcs not containing 
circuits s.t. each node of $D$ is entered at most by one arc in $A'$. 
(So the arc in $A'$ make up a forest). % A or A' ???

~

\splitpage{
\insertTikzFigure{1}{branchingExemple_base}
{A digraph $D=(A, V)$}
{tikz:branchingExemple_base}
}{
\insertTikzFigure{1}{branchingExemple_simple}
{A branching of $D$ routed at $a$.}
{tikz:branchingExemple_simple}
}
\end{frame}

%*************************************************************
\begin{frame}
\frametitle{Edge disjooint branching}

\splitpage{
    \insertTikzFigure{1}{branchingExemple_disjoint}
    {Two edge-disjoint branching routed respectively at $a$ and $d$.}
    {tikz:branchingExemple_disjoint}
}{
    \insertTikzFigure{1}{branchingExemple_arouted}
    {Two edge-disjoint $a$-routed branching.}
    {tikz:branchingExemple_arouted}
}
\end{frame}

%*************************************************************
\begin{frame}
\frametitle{a-cut}

\textbf{a-cut :}
A $a-$cut of $G$ determined by a set $S \subset V(G)$ is the set of
edges going from $S$ to $V(G) - S$. 
It will be denoted by $\Delta_G(S)$.
We also set that $\delta_G(S) = |\Delta_G(S)|$.

~

\threesplitpage
{
    \insertTikzFigure{0.6}{cutExemple_cut1}
    {A a-cut of $D$,\\ $S = \{a\}$,\\ $\Delta_D(S)=\{ab, ad\}$,\\ $\delta_D(S) = 2$.}
    {tikz:cutExemple_cut1}
}{
    \insertTikzFigure{0.6}{cutExemple_cut2}
    {A minimal a-cut of $D$,\\ $S = \{a, b, c, d, f\}$,\\ $\Delta_D(S)=\{be\}$,\\ $\delta_D(S) = 1$.}
    {tikz:cutExemple_cut2}
}{
    \insertTikzFigure{0.6}{cutExemple_cut3}
    {A minimal a-cut of $D$,\\ $S = \{a, b, d, e, f\}$,\\ $\Delta_D(S)=\{bc\}$,\\ $\delta_D(S) = 1$.}
    {tikz:cutExemple_cut3}
}
\end{frame}

%-------------------------------------------------------------
\subsection{Theorem}
%*************************************************************
\begin{frame}
\frametitle{Theorem}

\begin{edmonds_theorem}[Edmonds]
The maximum number of edge-disjoint branchings (rooted at $a$) 
equals the minimum number of edges in $a-$cuts.
\end{edmonds_theorem}
\end{frame}

%-------------------------------------------------------------
\subsection{Proof}
%*************************************************************
\begin{frame}
\frametitle{Proof}

Let $k$ represent the number of edges. If $\Delta_G(S) \ge k$ $\forall$ $S\subset V(G), a\in S$ then there are $k$ edge-disjoint branchings. 

Let $F$ be a set of edges such that $F$ is an arborescence rooted at $a$. Then we want to show that $\Delta_{G-F}(S) \ge k-1$ $\forall$ $S\subset V(G),$ $a\in S$.

If $F=E(G)$ then it is a branching and we are done, else $G-F$ contains $k-1$ edge-disjoint branchings and $F$ is in the $kth$ edge-disjoint branching. Following from this, we can see that $F$ only covers a set of vertices $T \subset V(G)$. Therefore, there exist a vertex $v$ that is not connected to $F$ and we can add an edge $e\in \Delta_G(T)$. This will yield an arborescence $F+e$ such that $F$ covers all points of $G$, satisfying $\Delta_{G-F}(S) \ge k-1$ $\forall$ $S\subset V(G),$ $a\in S$.
\end{frame}

%*************************************************************
\begin{frame}
\frametitle{Proof}
\insertTikzFigure{0.6}{p1edgeexist}
{}
{tikz:p1edgeexist}

If we have a maximal set $A\subset V(G)$ such that $A$ contains the root $a$, $A \cup T \ne V(G)$ and $\delta_{G-F}(S) = k-1$. If no such $A$ exist then we have set $A$ which is not maximal. Then we have $\delta_{G-F}(A\cup T) > \delta_{G-F}(A)$ that we can add a new edge $e=(x,y)$ which belongs to $\Delta_{G-F}(A\cup T) - \Delta_{G-F}(A)$ such that $x\in T-A$ and $y\in V(G) -T-A$.
\end{frame}

%*************************************************************
\begin{frame}
Now we want to show that that we can add $e$ to $F$ and $F+e$ will be an arborescence and satisfy $\Delta_{G-F}(S) \ge k-1$. For sure, we can see that $F$ is an arborescence. If $e\not\in \Delta_G(S)$ then $$\delta_{G-F-e}(S) = \delta_{G-F}(S) \ge k-1$$
Now, if we have that $e\in \Delta_G(S)$ then $x\in S$ and $y\in V(G)-S$. That is, $x$ and $y$ are in different set of vertices of $G$. Next, we will use the inequality; 
\begin{equation}\delta_{G-F}(S\cup A) + \delta_{G-F}(S\cap A) \le \delta_{G-F}(S) + \delta_{G-F}(A)\end{equation}
We already know from the maximality of $A$ that $\delta_{G-F}(A)=k-1$, $\delta_{G-F}(S\cap A) \ge k-1$ and $\delta_{G-F}(S\cup A) \ge k$. Then from $(1)$ we have $\delta_{G-F}(S) \ge k$. Therefore $\delta_{G-F-e}(S) \ge k-1$. 
\eofproof
\end{frame}

%=============================================================
\section{Second theorem}
%-------------------------------------------------------------
\subsection{Definitions}
%*************************************************************
\begin{frame}
\frametitle{Definitions}

\textbf{directed cut (dicuts) :}
A directed cut of a weakly connect graph $G$ is the set $D = \Delta_G(S),\ (S \subset V(G),\ S \neq \varnothing)$ provided $\Delta_G(S),\ (V(G) - S) = \varnothing$.

~
    
\textbf{dijoin :} A dijoin is a subset $A' \subset A$ which covers all dicuts. Let $\Omega(D)$ denote the maximum number of arc-disjoint dicuts in $D$ 
and let $\epsilon(D)$ be the minimum cardinality of a dijoin.

\threesplitpage
{
    \insertTikzFigure{0.8}{dicuts_0}
    {A weakly connected digraph containing 3 dicuts}
    {tikz:dicuts_0}
}{
    \insertTikzFigure{0.8}{dicuts_1}
    {Arc-disjoint dicuts\\$\Omega(G) = 2$}
    {tikz:dicuts_1}
}{
    \insertTikzFigure{0.8}{dicuts_2}
    {Dijoin of minimal cardinality\\$\epsilon(G) = 2$}
    {tikz:dicuts_2}
}

\end{frame}

%-------------------------------------------------------------
\subsection{Theroem}
%*************************************************************
\begin{frame}
\frametitle{Theorem}
    
\begin{Lucchesi_and_Younger}[Lucchesi and Younger]
The maximum number of disjoint directed cuts equals the minimum number of edges which cover all directed cuts ($\Omega(G) = \epsilon(G)$).
\end{Lucchesi_and_Younger}

\end{frame}

%*************************************************************
\begin{frame}
We will use induction on the number of edges. \\
\textbf{Case 1:} If there are no edges, then the theorem is proved. \\
\textbf{Case 2:} Let $k$ be the maximum directed cuts in $G$, $G_e"$ a digraph resulting from contraction of $e$. If $G_e"$ contains at most $k-1$ edge-disjoint directed cuts then, by the induction hypothesis, $\exists$ $k-1$ edges $e_1,\dots,e_{k-1}$ covering all directed cuts of $G_e"$. \\
From above, we observe that adding $e$ yields $e, e_1,\dots,e_{k-1}$ which is $k$ edges and cover all directed cuts of $G$. Since we need at least $k$ edges, the assertion is proved and we may assume that $G_e"$ contains $k$ disjoint directed cuts for each edge $e$.\\

If we subdivide all the edges of $G$ by a point, then the graph that we will obtain contains $k+1$ disjoint directed cuts. Hence we can find a subdivision $H$ of $G$ that contains $k$ at most $k$ disjoint directed cuts. If we again subdivide an edge $f$ of $H$ by a point, then it will contain $k+1$ disjoint directed cuts. Hence $H$ contains $k+1$ directed cuts $D_1,\dots,D_{k+1}$ such that only two of them have $f$ as a common edge. 

From above, $H_"f$ arises from either $G$ or $G_f"$ by subdivision. Hence from the assumption made above, $H$ contains $k$ disjoint directed cuts $c_1,\dots,c_k$ and $f$ not in $c_i$. Thus $D_1,\dots,D_{k+1},c_1,\dots,c_k$ is a collection of directed cuts of $G_0$ and any edge belongs to at most two of them. 

\end{frame}

%*************************************************************
\begin{frame}

Next, we have the lemma:\\
\begin{ly_lemma}[]
If a digraph $G$ contains at most $k$ disjoint directed cuts, and $F$ is any collection of directed cuts in $G$ such that any edge belongs to at most two of them, then $|F| \le 2k$. 
\end{ly_lemma}

% \begin{proof}[Proof]
Let $D=\Delta_G(S)$, $(S\subset V(G), S\ne \emptyset$) provided $\Delta_G(V(G)-S)=\emptyset$. Let $S_D$ be a set uniquely determined by by a directed cut $D$\\

First we replace $F$ by a collection of a simple structure. Let $D_1, D_2 \in F$ be called a laminar if $S_{D_1}\cap S_{D_2}=\emptyset \text{ or } S_{D_1}\subset S_{D_2}, \text{ or } S_{D_2}\subset S_{D_1} \text{ or } S_{D_1}\cup S_{D_2} = V(G)$. Else $D_1, D_2$ is a crossing\\

Let $D_1, D_2$ be a crossing pair and set the following:
$$D_1'=\Delta_G(S_{D_1} \cup S_{D_2})$$
$$D_2'=\Delta_G(S_{D_1} \cap S_{D_2})$$
$$F'=F\cup \{D_1',D_2' \}-\{D_1,D_2 \}$$

From these, we can see that $D_1', D_2'$ are directed cuts and cover any edge the same number of times as $D_1, D_2$. Hence $|F'| = |F|$. 

\end{frame}

%*************************************************************
\begin{frame}

Also,
$$\underset{D\in F}{\sum}|S_D|^2 < \underset{D\in F'}{\sum}|S_D|^2$$
since 
$$|S_{D_1}\cup S_{D_2}|^2 + |S_{D_1}\cap S_{D_2}|>|S_{D_1}|^2+|S_{D_2}|^2$$

Hence, we cannot repeat the same process of $F$ for $F'$ and achieve a cycle. We get a collection $F_0$ of directed cuts such that any two are a laminar and $|F_0| = |F|$. 

Next, we prove the \textit{Lemma} for the case when $F$ consist of pairwise laminar cuts. 

Let $F=\{D_1,\dots,D_N\}$, then we construct a graph $G'$ as follows:

$V(G')=\{v_1,\dots,v_N\}$ and join $v_i$ to $v_j$ \textit{iff} $D_i \cap D_j \ne \emptyset$. Then $G'$ contains at most $k$ independent points. 

\end{frame}

%*************************************************************
\begin{frame}

Now we will show that $G'$ is bipartite. \\
Consider a circuit $(v_1,\dots,v_m)$ in $G'$ and the corresponding sets $S_{D_1},\dots,S_{D_m}$. Then $D_1,\dots,D_m$ must be different. If $D_v = D_\mu$ then each edge of $D_v$ also belongs to $D_\mu$ and no other member of $F$. Hence $v$ has degree 1 and cannot occur in any circuit of $G'$.

Since $D_i \cap D_{i+1} \ne \emptyset$, we have either $S_{D_i}\subset S_{D_{i+1}}$ or $S_{D_{i+1}}\subset S_{D_i}$

We claim that both possibilities occur alternatively and this will prove that $m$ is even. Suppose not, example, $S_{D_0} \subset S_{D_1}\subset S_{D_2}$. 

We say $D_i$ is to the left from $D_j$ if either $S_{D_i}\subset S_{D_j}$ or $V(G)-S_{D_i}\subset S_{D_j}$. 

$D_i$ is to the right from $D_j$ if $S_{D_i}\subset V(G)-S_{D_j}$ or $V(G)-S_{D_i}\subset V(G)-S_{D_j}$. 

Now, since $F$ consist of laminar cuts, each $D_i\ne D_j$ is either to the left or to the right from $D_j$.

Since $D_2$ is to the right from $D_1$ but $D_0 = D_m$ is to the left from $D_1$, there is a $j$, $1\le j\le m-1$ such that $D_j$ is to the right from $D_1$ but $D_{j+1}$ is to the left from $D_1$.  
    
Since $D_j$ and $D_{j+1}$ have a common edge $e$, which must belong to $D_1$. Hence $e$ belongs to three cuts, which is a contradiction. 

\eofproof
\end{frame}

%*************************************************************
\begin{frame}

\textbf{Definitions : }
A \textit{hypergraph} H is a finite collection of finite sets. These sets are called edges and the elements of the edges are called vertices. Let $V(H)$ denote the set of vertices. If $E_1,\dots,E_m$ are the edges and $v_1,\dots,v_n$ are the vertices of H, then we define 
$$
    a_{ij}=\left\{
    \begin{array}{@{}ll@{}}
    1, & \text{if}\ v_i \in E_j \\
    0, & \text{otherwise}
    \end{array}\right.
$$

\begin{hypergraph}[Hypergraph]
If any hypergraph H' arising from H by multiplication of the vertices satisfies $v_2(H') = 2v(H')$ then $\tau(H) = v(H)$. 
\end{hypergraph}

\begin{proof}
First we show that\\
\textbf{1.} If $F$ is a collection of pairwise laminar directed cut then its incidence matrix $A$ is totally unimodular. 

\end{proof}
\end{frame}


\end{document}

