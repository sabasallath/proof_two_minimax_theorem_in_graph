\documentclass[8pt]{beamer}
\usepackage{packages}
\usepackage{commands}
\graphicspath{{img/}}
% !TeX root = ./on_two_minimax_theorems_in_graph_presentation.tex
% !TEX program = xelatex


%*************************************************************
% BEAMER THEME AND TWIX
%*************************************************************
\setbeamertemplate{bibliography item}{\insertbiblabel}
\usetheme[progressbar=frametitle]{metropolis}
\useoutertheme{metropolis}
\useinnertheme{metropolis}

\setbeamertemplate{mini frame}{}
\setbeamertemplate{mini frame in current section}{}
\setbeamertemplate{mini frame in current subsection}{}
\setbeamercolor{section in head/foot}{fg=normal text.bg, bg=mDarkTeal}
\setbeamercolor{subsection in head/foot}{fg=normal text.bg, bg=structure.fg}

\setbeamercolor{normal text}{fg=black}
\setbeamercolor{frametitle}{bg=mDarkTeal}

\makeatletter
\setlength{\metropolis@progressinheadfoot@linewidth}{2pt}

\setbeamertemplate{footline}{%
    \begin{beamercolorbox}[colsep=1.5pt]{upper separation line head}
    \end{beamercolorbox}
    \begin{beamercolorbox}{section in head/foot}
      \insertsubsectionnavigationhorizontal{\paperwidth}{}{\hskip0pt plus1filll}\vskip3pt%
    \end{beamercolorbox}%
    \begin{beamercolorbox}[colsep=1.5pt]{lower separation line head}
    \end{beamercolorbox}
}
\makeatother

%\setbeamercolor{background canvas}{bg=white}
%*************************************************************
\title{On Two Minimax Theorems in Graph}
\author{Daniel Blévin - George Apaaboah}
\date\today

%*************************************************************
\begin{document}

\maketitle

\begin{frame}
\tableofcontents
\end{frame}

%************************************************************* THEOREM
\newtheorem{edmonds_theorem}{THEOREM}
\newtheorem{Lucchesi_and_Younger}{THEOREM}

%************************************************************* 2
\section{First theorem}

% \subsection*{Branching and a-cut (from books)}
% \paragraph{branching}
% A branching $B$ is an arc set in a digraph $D$ that is a forest such that every node of $D$ 
% is the head of at most one arc of $B$. 
% A branching that is a tree is called an arborescence.
% A branching that is a spanning tree of $D$ is called a spanning arborescence of $D$. 
% Clearly, in a spanning arborescence B of $D$ every node of $D$ is the head of one arc of $B$ 
% except for one node. 
% This node is called the root of $B$. 
% If $r$ is the root of arborescence $B$ we also say that $B$ is rooted at $r$ or $r-$rooted.

% \paragraph{r-arborescence}
% An arborescence in a digraph $D = (V, A)$ is a set $A'$ of arcs making up a spanning tree 
% such that each node of $D$ is entered by at most one arc in $A'$. 
% It follows that there is exactly one node r that is not
% entered by any arc of $A'$. This node is called the root of $A'$, and $A'$ is called
% rooted in $r$, or an $r-$arborescence.

% \paragraph{r-cut}
% An r-cut in a digraph is an edge set $\delta^+(X)$ for some $X \subset V(G)$ with $r \in X$.

% \paragraph{edge-disjoint branchings}
% Two branching are edge-disjoint if they do not have any internal edge in common.

% \paragraph{A note on orientation}
% A rooted tree is a tree in which one vertex has been designated the root.
% The edges of a rooted tree can be assigned a natural orientation, either away from 
% or towards the root, in which case the structure becomes a directed rooted tree.
% When a directed rooted tree has an orientation away from the root,
% it is called an arborescence, branching, or out-tree;
% when it has an orientation towards the root, 
% it is called an anti-arborescence or in-tree.

% %************************************************************* 2
\subsection{Definitions}

\begin{frame}
\frametitle{Branching}

\textbf{branching (r-arborescence) :}
In a directed graph $D = (V, A)$ a branching is a set of arcs not containing 
circuits s.t. each node of $D$ is entered at most by one arc in $A'$. 
(So the arc in $A'$ make up a forest). % A or A' ???

~

\splitpage{
\insertTikzFigure{1}{branchingExemple_base}
{A digraph $D=(A, V)$}
{tikz:branchingExemple_base}
}{
\insertTikzFigure{1}{branchingExemple_simple}
{A branching of $D$ routed at $a$.}
{tikz:branchingExemple_simple}
}

\end{frame}

%*************************************************************
\begin{frame}
\frametitle{Edge disjooint branching}

\splitpage{
    \insertTikzFigure{1}{branchingExemple_disjoint}
    {Two edge-disjoint branching routed respectively at $a$ and $d$.}
    {tikz:branchingExemple_disjoint}
}{
    \insertTikzFigure{1}{branchingExemple_arouted}
    {Two edge-disjoint $a$-routed branching.}
    {tikz:branchingExemple_arouted}
}

\end{frame}

%*************************************************************
\begin{frame}
\frametitle{a-cut}

\textbf{a-cut :}
A $a-$cut of $G$ determined by a set $S \subset V(G)$ is the set of
edges going from $S$ to $V(G) - S$. 
It will be denoted by $\Delta_G(S)$.
We also set that $\delta_G(S) = |\Delta_G(S)|$.

~

\threesplitpage
{
    \insertTikzFigure{0.6}{cutExemple_cut1}
    {A a-cut of $D$,\\ $S = \{a\}$,\\ $\Delta_D(S)=\{ab, ad\}$,\\ $\delta_D(S) = 2$.}
    {tikz:cutExemple_cut1}
}{
    \insertTikzFigure{0.6}{cutExemple_cut2}
    {A minimal a-cut of $D$,\\ $S = \{a, b, c, d, f\}$,\\ $\Delta_D(S)=\{be\}$,\\ $\delta_D(S) = 1$.}
    {tikz:cutExemple_cut2}
}{
    \insertTikzFigure{0.6}{cutExemple_cut3}
    {A minimal a-cut of $D$,\\ $S = \{a, b, d, e, f\}$,\\ $\Delta_D(S)=\{bc\}$,\\ $\delta_D(S) = 1$.}
    {tikz:cutExemple_cut3}
}

\end{frame}

%*************************************************************
\subsection{Theorem}
\begin{frame}
\frametitle{Theorem}

\begin{edmonds_theorem}[Edmonds]
The maximum number of edge-disjoint branchings (rooted at $a$) 
equals the minimum number of edges in $a-$cuts.
\end{edmonds_theorem}

\end{frame}

%*************************************************************
\subsection{Proof}
\begin{frame}
\frametitle{Proof}

Let $k$ represent the number of edges. If $\Delta_G(S) \ge k$ $\forall$ $S\subset V(G), a\in S$ then there are $k$ edge-disjoint branchings. 

Let $F$ be a set of edges such that $F$ is an arborescence rooted at $a$. Then we want to show that $\Delta_{G-F}(S) \ge k-1$ $\forall$ $S\subset V(G),$ $a\in S$.

If $F=E(G)$ then it is a branching and we are done, else $G-F$ contains $k-1$ edge-disjoint branchings and $F$ is in the $kth$ edge-disjoint branching. Following from this, we can see that $F$ only covers a set of vertices $T \subset V(G)$. Therefore, there exist a vertex $v$ that is not connected to $F$ and we can add an edge $e\in \Delta_G(T)$. This will yield an arborescence $F+e$ such that $F$ covers all points of $G$, satisfying $\Delta_{G-F}(S) \ge k-1$ $\forall$ $S\subset V(G),$ $a\in S$.

\end{frame}
\begin{frame}
\frametitle{Proof}

If we have a maximal set $A\subset V(G)$ such that $A$ contains the root $a$, $A \cup T \ne V(G)$ and $\delta_{G-F}(S) = k-1$. If no such $A$ exist then we have set $A$ which is not maximal. Then we have $\delta_{G-F}(A\cup T) > \delta_{G-F}(A)$ that we can add a new edge $e=(x,y)$ which belongs to $\Delta_{G-F}(A\cup T) - \Delta_{G-F}(A)$ such that $x\in T-A$ and $y\in V(G) -T-A$.

Now we want to show that that we can add $e$ to $F$ and $F+e$ will be an arborescence and satisfy $\Delta_{G-F}(S) \ge k-1$. For sure, we can see that $F$ is an arborescence. If $e\not\in \Delta_G(S)$ then $$\delta_{G-F-e}(S) = \delta_{G-F}(S) \ge k-1$$
Now, if we have that $e\in \Delta_G(S)$ then $x\in S$ and $y\in V(G)-S$. That is, $x$ and $y$ are in different set of vertices of $G$. Next, we will use the inequality; 
\begin{equation}\delta_{G-F}(S\cup A) + \delta_{G-F}(S\cap A) \le \delta_{G-F}(S) + \delta_{G-F}(A)\end{equation}
We already know from the maximality of $A$ that $\delta_{G-F}(A)=k-1$, $\delta_{G-F}(S\cap A) \ge k-1$ and $\delta_{G-F}(S\cup A) \ge k$. Then from $(1)$ we have $\delta_{G-F}(S) \ge k$. Therefore $\delta_{G-F-e}(S) \ge k-1$. 

\end{frame}


\section{Second theorem}
\subsection{Definitions}
\subsection{Theroem}

%*************************************************************
\begin{frame}
\frametitle{Directed cut}

\textbf{directed cut (dicuts) :}
A directed cut of a weakly connect graph $G$ is the set $D = \Delta_G(S),\ (S \subset V(G),\ S \neq \varnothing)$ provided $\Delta_G(S),\ (V(G) - S) = \varnothing$. 
%It's a set of arcs of the form $(X, V - X)$, where $X$ is a non-empty proper subset of $V$ such that there are no arcs from $V - X$ to $X$.

\end{frame}

%*************************************************************
\begin{frame}
\frametitle{dijoin}
    
\textbf{dijoin :} A dijoin is a subset $A' \subset A$ which covers all dicuts.

~

Let $\Omega(D)$ denote the maximum number of arc-disjoint dicuts in $D$ 
and let $\epsilon(D)$ be the minimum cardinality of a dijoin.

\end{frame}

%*************************************************************
\begin{frame}
\frametitle{Theorem}
    
\begin{Lucchesi_and_Younger}[Lucchesi and Younger]
The maximum number of disjoint directed cuts equals the minimum number of edges which cover all directed cuts.
\end{Lucchesi_and_Younger}

\end{frame}


\end{document}

